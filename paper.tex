\documentclass[11pt]{article} \usepackage[a4paper,margin=1in]{geometry} \usepackage{graphicx} \usepackage{amsmath} \usepackage{hyperref} \usepackage{booktabs} \usepackage{enumitem} \usepackage{float} \usepackage{setspace} \title{\textbf{Governed vs Autonomous AI Coding:\\ Structural Drift, Architectural Entropy, and Empirical Evidence from Three Longitudinal Case Studies}} \author{Independent Research Study} \date{\today} \begin{document} \maketitle \onehalfspacing \begin{abstract} This companion study presents empirical longitudinal evidence supporting the EntropyX framework for measuring structural drift in software repositories. Three real-world repositories were analyzed under varying development regimes: autonomous AI generation, human-led refactoring, and hybrid AI-assisted development. The results show that fully autonomous AI phases are associated with rapid accumulation of entropy, whereas human architectural intervention yields a measurable reduction in entropy. Hybrid AI-assisted models remain structurally stable when architecture governance remains human-controlled. The size and complexity of the repository act as risk multipliers. EntropyX is proposed not as a quality grade, but as a structural temperature indicator that enables intentional architectural governance. \end{abstract} \section{Introduction} AI coding systems have evolved from completion tools to autonomous agents capable of generating entire subsystems. Although productivity gains are evident, structural maintainability risk remains under-examined. EntropyX was developed to measure structural drift longitudinally throughout the commit history. This study applies EntropyX to three repositories: \begin{itemize} \item ShaderLabDX12 \item DrCircuitsCanvasLibrary \item shaderforge \end{itemize} Each contains distinct development regimes, enabling comparative analysis. \section{EntropyX Model Overview} For each file $i$: \begin{equation} b_i = \log(1 + SLOC_i) + \alpha \cdot CC_i + \beta \cdot Smell_i \end{equation} The distribution is normalized: \begin{equation} p_i = \frac{b_i}{\sum_j b_j} \end{equation} Shannon diffusion: \begin{equation} H = - \sum p_i \log(p_i) \end{equation} EntropyX combines magnitude and normalized diffusion: \begin{equation} EntropyX = \overline{b} \cdot H_{norm} \end{equation} EntropyX measures structural temperature, not quality. \section{Diffusion Contribution and Structural Localization} EntropyX decomposes the diffusion per file: \begin{equation} contrib_i = -p_i \log(p_i) \end{equation} This enables identification of: \begin{itemize} \item Dominant structural hotspots \item Diffusion spreaders \item Delta contributors across commits \end{itemize} Thus, EntropyX becomes actionable rather than descriptive. \section{Case Study I: ShaderLabDX12} Autonomous AI generation followed by major human refactor. \begin{figure}[H] \centering \includegraphics[width=0.9\textwidth]{figures/shaderlab_entropy.svg} \caption{EntropyX longitudinal evolution for ShaderLabDX12. Autonomous AI phase exhibits rapid entropy growth followed by substantial reduction during human-led architectural refactor.} \label{fig:shaderlab_entropy} \end{figure} \begin{figure}[H] \centering \includegraphics[width=0.9\textwidth]{figures/shaderlab_sloc.svg} \caption{SLOC growth curve for ShaderLabDX12. Sudden SLOC expansion correlates with entropy shock events.} \label{fig:shaderlab_sloc} \end{figure} Findings \begin{itemize} \item Autonomous phase: steep slope of entropy. \item Human refactor: immediate cooling of entropy. \item Hybrid governance: growth of bounded entropy. \end{itemize} \section{Case Study II: DrCircuitsCanvasLibrary} Originally developed by humans. AI was introduced later to implement a TypeScript interface. \begin{figure}[H] \centering \includegraphics[width=0.9\textwidth]{figures/canvas_entropy.svg} \caption{EntropyX evolution for DrCircuitsCanvasLibrary. Major 2018 rewrite followed by stabilization cooling. AI-generated TypeScript adapter produces significant entropy spike.} \label{fig:canvas_entropy} \end{figure} Key observation: The AI-generated adapter introduced greater diffusion relative to system size than the earlier engine rewrite. \section{Case Study III: shaderforge} AI-generated GUI/API abandoned. Complete human-governed rewrite performed. Later, AI was used only as an implementation assistant. \begin{figure}[H] \centering \includegraphics[width=0.9\textwidth]{figures/shaderforge_entropy.svg} \caption{EntropyX evolution for shaderforge. Autonomous plateau at high entropy followed by reset during greenfield rewrite. Hybrid phase remains structurally bounded.} \label{fig:shaderforge_entropy} \end{figure} The rewrite significantly reduced the entropy despite the expansion of the system. \section{Monolith Decomposition and Diffusion Effects} When a large monolithic file is decomposed: \begin{itemize} \item The magnitude of the file decreases. \item The distribution becomes flatter. \item Shannon entropy increases. \end{itemize} This is expected behavior, not model failure. EntropyX captures the trade-off between centralized complexity and distributed complexity. The distributed architecture increases diffusion; whether this represents fragmentation or modularization depends on the coupling context. Future refinements may incorporate coupling-weighted diffusion. \section{Repository Complexity as Risk Multiplier} Small or greenfield systems: \begin{itemize} \item Easier global reasoning \item Controlled diffusion \item Lower context fragmentation risk \end{itemize} Large repositories: \begin{itemize} \item SLOC shock events amplify the entropy \item Context window limitations restrict global coherence \item Diffusion increases cognitive overhead \end{itemize} Repository complexity moderates AI impact. \section{Intent vs Automation} A key empirical distinction emerged: \begin{itemize} \item Human-led changes include architectural intent. \item Autonomous AI changes increase the entropy without an embedded narrative. \end{itemize} EntropyX enables post-hoc localization of drift: \begin{itemize} \item Where did entropy increase? \item Which files drove the diffusion? \item Is magnitude or distribution dominant? \end{itemize} This allows engineering effort to be directed intentionally. \section{Governance Model} EntropyX supports an operational workflow: \subsection*{If Entropy Rises} \begin{enumerate} \item Identify the affected view (Production vs. utility vs. full). \item Inspect delta contributors. \item Determine diffusion vs magnitude dominance. \item Decide whether to refactor, modularize, consolidate, or stabilize. \end{enumerate} \subsection*{If Entropy Falls} \begin{enumerate} \item Identify cooling patterns. \item Capture architectural decisions that reduce diffusion. \item Institutionalize stabilizing practices. \end{enumerate} EntropyX prioritizes attention rather than replacing judgment. \section{Ethical Considerations} AI systems are probabilistic amplifiers. Errors are inevitable. The ethical responsibility shifts from preventing all errors to: \begin{itemize} \item Measurement of structural risk. \item Monitoring drift. \item Governing integration intentionally. \end{itemize} Blind automation increases the risk of accumulation of entropy. Governed usage enables productivity without surrendering architectural accountability. \section{Conclusion} Across all three repositories: \begin{itemize} \item Autonomous AI coding correlates with rapid entropy growth. \item The human architectural intervention measurably reduces entropy. \item Hybrid governance yields sustainable structural evolution. \item The complexity of the repository increases the risk. \end{itemize} EntropyX functions as a structural temperature instrument, enabling intentional architectural governance in AI-assisted development environments. Future work includes larger cross-team studies and coupling-aware diffusion refinement. \end{document}